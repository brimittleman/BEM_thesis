\abstract

Differences in gene regulation contribute to phenotypic differences within humans and also between humans and other primates. While co-transcriptional gene regulatory mechanisms such as alternative polyadenylation \emph{(APA)} can help explain how variation in gene regulation manifests, such mechanisms remain understudied. In this thesis, I used a quantitative genomics approach and a comparative primate approach to understand the regulatory role of APA. In chapter 2, I measured polyadenylation site \emph{(PAS)} usage genome wide in a population of 52 human lymphoblastoid cell lines. I identified genetic variation associated with APA \emph{(apaQTLs)} and showed that genetic variation acts through PAS choice to impact mRNA expression, translation, and protein levels in complex, non-linear ways. In chapter 3, I measured APA conservation between human and chimpanzee. While APA is largely conserved, differences in PAS usage and isoform diversity contribute to differentially expressed and differentially translated genes. Together, these chapters, establish APA as a key co-transcriptional mechanism underlying the genetic regulation of gene and protein expression levels. As a step to further understand co-transcriptional regulatory mechanisms, in chapter 4, I describe an attempt to measure polymerase II elongation rate genome wide. In the final chapter, I outline a set of necessary future directions to extend my work on APA to more tissues and biological processes.
\\
\\
 (Note:
Supplementary tables are provided in a .zip file available
online. Captions for the tables are provided within the dissertation.)
