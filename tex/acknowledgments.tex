\acknowledgments

The following dissertation would not have been possible without the immense amount of support that I have had throughout my time at the University of Chicago. It is difficult to put into words the gratitude that I have for all of the people that I have met here in Chicago. 

First, I would like to thank my advisor, Dr. Yoav Gilad. Yoav has continued to be my advocate scientifically and personally. I am thankful both for how he has pushed me to become a better scientist while also providing me the intellectual freedom to pursue my goals. I hope that in the future I can continue to call Yoav both a mentor and a friend. I was also lucky to work with Dr. Yang Li as he started his lab at University of Chicago. Working with him and members of his lab has taught me about being a collaborator, mentor, and mentee. I am also forever grateful to have Dr. Sebastian Pott as an additional mentor in my lab. Seb is both a thoughtful scientist and a great mentor. On a day to day basis, I knew I could count on Seb for both scientific and person advice. I would like to thank my other committee members, Dr. Matthew Stephens and Dr. Jon Staley. They both provided an outside prospective and helped me to think about my work from a different angle. 

From my first days at University of Chicago, the Gilad lab has been my home. Each and every member of the Gilad lab has made this experience worth it! I am so lucky to have started in the lab with 2 amazing women, Katie Rhodes and Reem Elorbany. While we cannot be any more different as people, our friendships are ones I will cherish forever. We will forever be the 'cohort' and remember to work 'deep in the hood'. Lauren Blake deserves a special thank you because she acted as my mentor from the day I interviewed until the day I defended. I learned more on our walks home than I did in many classes! I am so glad we will both be in California for the next few years. Thank you to John Blischak for teaching me the importance of reproducible science and ensuring that I learned and implemented best practices for data science as soon as I rotated in the Stephens lab. I also have to thank John for helping me get involved with Software Carpentry.  Each Gilad lab member past and present, has influenced my scientific and personal views. The other graduate students, Bryan Pavlovic, John Blischak, Lauren Blake, Ittai Eres, Katie Rhodes, Reem Elorbany, Anthony Hung, Deji Adegunsoye, Wenhe Lin and Erik McIntire. The post-docs Po-Yuan Tung, Joyce Hsiao, Michelle Ward, Genevieve Housman, Kenneth Barr, Ben Fair, Ben Umans. The lab managers and technicians, Kristen Patterson, Jonathan Burnett, Claudia Cuevas, and Emilie Briscoe. Also special thanks to our science writer Natalia Gonzales. I have also had to great opportunity to work with members of the Li lab including graduate student, Phoenix Mu, and the undergrads Tony Zeng, and Shane Warland.

I have also made some wonderful friends in the GGSB and HG program. Among others, Sahar Mozaffari, Linsin Smith, Sammy Keyport, Ryan Dohn, Michael Drazer, and Manny Vazquez. Other members of the research community, Ezra Amiri, Edgar Correa, Tomasz Slezak, and Daniel Downie. Thank you to Helen Robertson for her friendship and for providing useful feedback on chapters 1 and 5 of this dissertation. I need to give a special thank you to Haley Randolph, who also became my only roommate in Chicago. Haley works harder than anyone I know but still has time to be a true friend! I know she will solve big problems in Genomics and Immunology one day and I plan to be her biggest fan. 
Sue Levison has been way more than just a graduate program administrator during my time in Chicago. Sue has been my biggest fan and friend from the moment I got to Chicago. I value her advice and emotional support over the last 4 years. 
Scientific outreach and community engagement helped me stay grounded during my Ph.D. I would like to thank Shaz Rasul, Monica Luna, and the rest of the Neighborhood Schools Program team for backing my SMART Science program. Also, thank you to the students who helped make sure the program has been and will continue to be successful. 

My success in this program would not have been possible without the communities outside of the University that have provided me a home in Chicago. First, Trapeze School New York in Chicago. Not only has trapeze given me amazing friends and a super unique hobby, it gave me a reason to branch out of my community and take a break each week. I will never forget the Sunday's that I spent working at the Edgewater Starbuck's then flying in my IFW. Second, Orange Theory Fitness Hyde Park became more than a gym to me over the last few years. Through the OTF HP community I was able to rediscover my love for fitness. There are many stressful lab days that would have been a lot harder if I did not have my OTF workout to de-stress! 

I made many lifetime friends as an undergraduate at Duke University. They have continued to be my biggest cheerleaders while I have been here in Chicago. First, thank you to my friends who lived in Chicago for at least a short time when I was here, Katie Heckman, Lauren Rosen, Adriana Dickerson, and Amanda Jones. I have to thank them for making the trek down to Hyde Park to come spend time with me sometimes! Also thank you to my friends who supported me from afar, Breanna Atkinson, Linda Zambrano, Sonia Lee. Also thank you the rest of my Duke Cheerleading class of 2016 family. Each of these people and so many more, mean so much to me and I am grateful that our friendships have stayed strong since graduation from Duke.

I would not be submitting this Ph.D. thesis without the continued support of Dr. Mohamed Noor. I will forever owe him for providing me the opportunity to start working in his lab. I am so lucky to have an undergraduate research advisor who will give me advice in hard times and be the first to congratulate me on achievements. 

Finally, I need to thank my family. Even from across the country my mom- Lisa, dad- David, sister- Ariel, and brother-Bradley have continued to be my rocks. While they never understood what I was studying they still were happy to hear about the highs and lows of my research. I am also extremely lucky to have 4 supportive grandparents, Grandma Nancie, Papa Len, Grandma Linda, and Grandpa Richard. They each had the opportunity to visit me in Chicago during my graduate work. I hope to continue to make my family proud in the years to come.
